\documentclass[SE]{lsstdoc}
\usepackage{enumitem}

\setDocChangeRecord{%
\addtohist{}{2018-01-24}{Initial version.} {Gregory Dubois Felsmann}
%\addtohist{1.0}{2017-12-12}{Approved in \jira{RFC-xxx}.}{W.~O'Mullane, T.~Jenness}
}

\title[Dat product categories  ]{New Nomenclature for the LSST Data Products}

\author   {LSST Project Office}
\setDocRef      {LPM-231} % the reference code
\setDocDate     {\today}              % the date of the issue
\setDocUpstreamLocation{\url{https://github.com/lsst/LPM-231}}
%
% a short abstract
%
\setDocAbstract {
This note defines the user-facing, operations-era language to be used for the three
main categories of data products to be served by LSST, and connects this language to
the construction-era language that is used in requirements and design documents.
}

\begin{document}
%
% the title page
%
\maketitle

\renewcommand{\thepage}{\arabic{page}}% Arabic numerals for page counter

\setcounter{page}{1}% Start page number
%\printnoidxglossaries
%
% It's all yours from here on
%
\input{body}

%\section{References\label{sect:references}}
%\renewcommand{\refname}{}
%\bibliography{local,lsst,lsst-dm,refs,books,refs_ads}

%\section{Acronyms}
%\input{acronyms} % generated by the acronyms.csh (GaiaTools)
\end{document}




